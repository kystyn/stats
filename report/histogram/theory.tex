\section{Теория}
\subsection{Распределения}
\begin{itemize}
\item{Нормальное распределение}
Плотность вероятности для нормального распределения (закона Гаусса):

\begin{equation}
  f(x, \mu, \sigma) = \frac{1}{{\sigma \sqrt {2\pi } }}e^-\frac{ \left( {x - \mu } \right)^2 } {2\sigma ^2 }
\end{equation}
 \\

При \( \sigma = 1 \) и \( \mu = 0 \):

\begin{equation}
  f(x, 0, 1) = \frac{1}{ \sqrt {2\pi } }e^{-\frac{{x}^2 }{2}}
\end{equation}


\item{Распределение Коши}
Плотность вероятности для распределения Коши:
$$
f(x, {x_0}, \gamma) =  \frac{1}{\pi\gamma \left[1 + \left(\frac{x-x_0}{\gamma}\right)^2\right]}
$$

При \( \gamma = 1 \) и \( x_0 = 0 \):
$$
f(x, 0, 1) =  \frac{1}{\pi \left[1 + {x}^2\right]}
$$


\item{Распределение Лапласа}
Плотность вероятности для распределения Лапласа:
\begin{equation} 
f(x, \beta, \alpha) = \frac{\alpha}{2} \, e^{-\alpha|x - \beta|}
\end{equation}

При \( \beta = 0 \) и \( \alpha = \frac{1}{\sqrt{2}} \):
\begin{equation} 
f(x, 0, \frac{1}{\sqrt{2}}) = \frac{1}{2\sqrt{2}} \, e^{-\frac{|x|}{\sqrt{2}}}
\end{equation}


\item{Распределение Пуассона}

Плотность вероятности для распределения Пуассона:
\begin{equation} 
f(k, \lambda) = \frac{\lambda^k}{k!}\, e^{-\lambda}
\end{equation}

При \( \lambda = 10 \):
\begin{equation} 
f(k, 10) = \frac{10^k}{k!}\, e^{-10}
\end{equation}

\item{Равномерное распределение}
Плотность вероятности для равномерного распределения:
\begin{equation} 
f(x, a, b) = \left\{
\begin{matrix}
{1 \over b-a}, & x\in [a,b] \\
0, & x\not\in [a,b]
\end{matrix}
\right..
\end{equation}

При \( a = -\sqrt{3} \) и \( b = \sqrt{3} \):
\begin{equation} 
f(x, -\sqrt{3}, \sqrt{3}) = \left\{
\begin{matrix}
{1 \over 2\sqrt{3}}, & x\in [-\sqrt{3},\sqrt{3}] \\
0, & x\not\in [-\sqrt{3},\sqrt{3}]
\end{matrix}
\right..
\end{equation}

\end{itemize}
