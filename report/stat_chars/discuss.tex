Во всех распределениях, кроме распределения Коши, можно отметить, что выборочное среднее и выборочная медиана располагаются друг другу ближе (в смысле модуля разности), чем к другим характеристикам. Также видно, что эти параметры положения являются наиболее стабильными: во всех распределениях, кроме распределения Коши, они обладают минимальной дисперсией по сравнению с другими характеристиками. Более того, вычисленные в разных экспериментах значения медианы распределения Коши также оказывались минимально разбросанными, из чего можно сделать вывод, что данную характеристику можно рекомендовать как лучший среди представленных параметров положения -- он показывает стабильный результат даже на таком специфичном распределении, как распределение Коши.

Аномальное поведение данных в распределении Коши, судя по всему, связано с отсутствием у него матожидания (в том смысле, что интеграл, позволяющий его посчитать, расходится).

Хочется отметить полусумму экстремальных значений как наименее стабильный параметр положения: его дисперсия всегда была в разы, а иногда и на несколько порядков, выше, чем у других характеристик. Оно и понятно: мы используем весьма специфичные и уникальные для каждого эксперимента точки -- выбросы распределения -- как опорные элементы для вычисления параметра положения, что, вообще говоря, нелогично.

Все остальные характеристики положения показывали достаточно схожие результаты и их стабильность.

Сравним характеристики для выборок из 1000 элементов по величине:
\begin{itemize}
	\item Нормальное распределение
	
	$\overline{x} \leq z_{tr} \leq med{x} \leq z_Q \leq z_R $
	
	\item Распределение Коши
	
	$z_R \leq \overline{x} \leq med{x} \leq z_{tr} \leq z_Q $
	
	\item Распределение Лапласа
	
	$ \overline{x} \leq z_{tr} \leq med{x} \leq z_Q \leq z_R $
	
	\item Распределение Пуассона
	
	$ z_{tr} \leq med{x} \leq z_Q \leq \overline{x} \leq z_R $
	
	\item Равномерное распределение
	
	$ z_R \leq z_{tr} \leq \overline{x} \leq med{X} \leq z_Q $
	
\end{itemize}
