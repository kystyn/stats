\section{Постановка задачи}
Требуется сгенерировать наборы случайных точек для пяти распределений:
\begin{itemize}
\item Нормальное распределение: $N(x, 0, 1)$
\item Распределение Коши: $C(x, 0, 1)$
\item Распределение Лапласа: $L(x, 0, \frac{1}{\sqrt{2}})$
\item Распределение Пуассона: $P(k, 10)$
\item Равномерное распределение: $U(x, -\sqrt{3}, \sqrt{3})$
\end{itemize}
Для каждого распределения наборы должны состоять из 10, 50, 1000 элементов. 

Требуется построить гистограммы для полученных данных вместе с графиками функций плотности.


\section{Теория}
\subsection{Нормальное распределение}
Плотность вероятности для нормального распределения (закона Гаусса):

\begin{equation}
  f(x, \mu, \sigma) = \frac{1}{{\sigma \sqrt {2\pi } }}e^-\frac{ \left( {x - \mu } \right)^2 } {2\sigma ^2 }
\end{equation}
 \\

При \( \sigma = 1 \) и \( \mu = 0 \):

\begin{equation}
  f(x, 0, 1) = \frac{1}{ \sqrt {2\pi } }e^{-\frac{{x}^2 }{2}}
\end{equation}


\subsection{Распределение Коши}
Плотность вероятности для распределения Коши:
$$
f(x, {x_0}, \gamma) =  \frac{1}{\pi\gamma \left[1 + \left(\frac{x-x_0}{\gamma}\right)^2\right]}
$$

При \( \gamma = 1 \) и \( x_0 = 0 \):
$$
f(x, 0, 1) =  \frac{1}{\pi \left[1 + {x}^2\right]}
$$


\subsection{Распределение Лапласа}
Плотность вероятности для распределения Лапласа:
\begin{equation} 
f(x, \beta, \alpha) = \frac{\alpha}{2} \, e^{-\alpha|x - \beta|}
\end{equation}

При \( \beta = 0 \) и \( \alpha = \frac{1}{\sqrt{2}} \):
\begin{equation} 
f(x, 0, \frac{1}{\sqrt{2}}) = \frac{1}{2\sqrt{2}} \, e^{-\frac{|x|}{\sqrt{2}}}
\end{equation}


\subsection{Распределение Пуассона}

Плотность вероятности для распределения Пуассона:
\begin{equation} 
f(k, \lambda) = \frac{\lambda^k}{k!}\, e^{-\lambda}
\end{equation}

При \( \lambda = 10 \):
\begin{equation} 
f(k, 10) = \frac{10^k}{k!}\, e^{-10}
\end{equation}

\subsection{Равномерное распределение}
Плотность вероятности для равномерного распределения:
\begin{equation} 
f(x, a, b) = \left\{
\begin{matrix}
{1 \over b-a}, & x\in [a,b] \\
0, & x\not\in [a,b]
\end{matrix}
\right..
\end{equation}

При \( a = -\sqrt{3} \) и \( b = \sqrt{3} \):
\begin{equation} 
f(x, -\sqrt{3}, \sqrt{3}) = \left\{
\begin{matrix}
{1 \over 2\sqrt{3}}, & x\in [-\sqrt{3},\sqrt{3}] \\
0, & x\not\in [-\sqrt{3},\sqrt{3}]
\end{matrix}
\right..
\end{equation}


\section{Реализация}
Данная работа реализована на языке программирования Python с использованием IDE PyCharm и библиотек MatPlotLib, NumPy, Seaborn в ОС Ubuntu 19.04.

Отчёт подготовлен с помощью компилятора pdflatex и среды разработки TeXworks.

\section{Результаты}
На нижележащих рисунках изображены гистограммы распределений и теоретические функции плотности распределения

\subsection{Нормальное распределение}
\begin{figure}[H]
	\begin{center}
		\includegraphics[scale=0.5]{norm_hist}
		\caption{Гистограмма и плотность вероятности для нормального распределения} 
		\label{pic:pic_name}
	\end{center}
\end{figure}

\subsection{Распределение Коши}
\begin{figure}[H]
	\begin{center}
		\includegraphics[scale=0.5]{cauchy_hist}
		\caption{Гистограмма и плотность вероятности для распределения Коши} 
		\label{pic:pic_name} 
	\end{center}
\end{figure}

\subsection{Распределение Лапласа}
\begin{figure}[H]
	\begin{center}
		\includegraphics[scale=0.5]{laplace_hist}
		\caption{Гистограмма и плотность вероятности для распределения Лапласа} 
		\label{pic:pic_name} 
	\end{center}
\end{figure}

\subsection{Распределение Пуассона}
\begin{figure}[H]
	\begin{center}
		\includegraphics[scale=0.5]{poisson_hist}
		\caption{Гистограмма и плотность вероятности для распределения Пуассона} 
		\label{pic:pic_name} 
	\end{center}
\end{figure}

\subsection{Равномерное распределение}
\begin{figure}[H]
	\begin{center}
		\includegraphics[scale=0.5]{uniform_hist}
		\caption{Гистограмма и плотность вероятности для равномерного распределения} 
		\label{pic:pic_name}
	\end{center}
\end{figure}


\section{Обсуждение}
Гистограммы в целом визуально повторяют графики плотности распределения, причём чем выше количество случайных величин, тем выше сходство теоретической функции плотности распределения и кривой, огибающей верхние точки столбцов гистограммы (суть, экспериментальной функции плотности распределения).

Однако существуют определённого рода различия: так, максимумы гистограмм и плотностей распределения почти нигде не совпали. Более того, соотношение, кто из них больше, может меняться от эксперимента к эксперименту (распределение Лапласа). Также наблюдаются всплески гистограмм, что наиболее хорошо прослеживается на распределении Коши.



\section{Литература}
Максимов Ю. Д. Математическая статистика //СПб.: СПбГПУ. – 2004.

\section{Приложения}

Репозиторий с кодом программы и кодом отчёта: \href{https://github.com/kystyn/stats}{https://github.com/kystyn/stats}



