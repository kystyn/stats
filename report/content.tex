\section{Постановка задачи}
Необходимо сгенерировать массивы данных для пяти распределений:
\begin{itemize}
\item нормальное распределение \( N(x, 0, 1) \)
\item распределение Коши \( C(x, 0, 1) \)
\item распределение Лапласа \( L(x, 0, 1/\sqrt{2}) \)
\item распределение Пуассона \( P(k, 10) \)
\item равномерное распределение \( U(x, -\sqrt{3}, \sqrt{3}) \)
\end{itemize}
Для каждого распределения массивы должны состоять из 10, 50, 1000 элементов. Построить гистограммы для полученных данных вместе с графиками функций плотности.



\section{Теория}
\subsection{Нормальное распределение}
Плотность вероятности для нормального распределения (закона Гаусса):

$
  f(x, \mu, \sigma) = \frac{1}{{\sigma \sqrt {2\pi } }}e^{{{ - \left( {x - \mu } \right)^2 } \mathord{\left/ {\vphantom {{ - \left( {x - \mu } \right)^2 } {2\sigma ^2 }}} \right. \kern-\nulldelimiterspace} {2\sigma ^2 }}}
$
 
В нашем частном случае при \( \sigma = 1 \) и \( \mu = 0 \):

$
  f(x, 0, 1) = \frac{1}{ \sqrt {2\pi } }e^{{{ -{x}^2 } \mathord{\left/ {\vphantom {{ -{x}^2 } {2\sigma ^2 }}} \right. \kern-\nulldelimiterspace} {2}}}
$


\subsection{Распределение Коши}
Плотность вероятности для распределения Коши:
$$
f(x, {x_0}, \gamma) =  \frac{1}{\pi\gamma \left[1 + \left(\frac{x-x_0}{\gamma}\right)^2\right]}
$$

Для частного случая при \( \gamma = 1 \) и \( x_0 = 0 \):
$$
f(x, 0, 1) =  \frac{1}{\pi \left[1 + {x}^2\right]}
$$


\subsection{Распределение Лапласа}
Плотность вероятности для распределения Лапласа:
\begin{equation} 
f(x, \beta, \alpha) = \frac{\alpha}{2} \, e^{-\alpha|x - \beta|}
\end{equation}

Для частного случая при \( \beta = 0 \) и \( \alpha = \frac{1}{\sqrt{2}} \):
\begin{equation} 
f(x, 0, \frac{1}{\sqrt{2}}) = \frac{1}{2\sqrt{2}} \, e^{-\frac{|x|}{\sqrt{2}}}
\end{equation}


\subsection{Распределение Пуассона}

Плотность вероятности для распределения Пуассона:
\begin{equation} 
f(k, \lambda) = \frac{\lambda^k}{k!}\, e^{-\lambda}
\end{equation}

Для частного случая при \( \lambda = 10 \):
\begin{equation} 
f(k, 10) = \frac{10^k}{k!}\, e^{-10}
\end{equation}

\subsection{Равномерное распределение}
Плотность вероятности для равномерного распределения:
\begin{equation} 
f(x, a, b) = \left\{
\begin{matrix}
{1 \over b-a}, & x\in [a,b] \\
0, & x\not\in [a,b]
\end{matrix}
\right..
\end{equation}

Для частного случая при \( a = -\sqrt{3} \) и \( b = \sqrt{3} \):
\begin{equation} 
f(x, -\sqrt{3}, \sqrt{3}) = \left\{
\begin{matrix}
{1 \over 2\sqrt{3}}, & x\in [-\sqrt{3},\sqrt{3}] \\
0, & x\not\in [-\sqrt{3},\sqrt{3}]
\end{matrix}
\right..
\end{equation}


\section{Реализация}
\begin{itemize}
\item Язык: Python
\item Среда разработки: PyCharm
\item Используемые библиотеки: NumPy, SciPy
\end{itemize}


\section{Результаты}
На следующих изображениях приведены графики плотности распределений (теоретические), гистограммы распределений сгенерированных значений, а также графики плотности распределений, полученных исходя из сгенерированных значений (темно зелёный график).

\subsection{Нормальное распределение}
\begin{figure}[H]
	\begin{center}
		\includegraphics[scale=0.7]{fig/Normal_distr.png}
		\caption{Экспериментальные и теоретические данные для нормального распределения} 
		\label{pic:pic_name} % название для ссылок внутри кода
	\end{center}
\end{figure}

\subsection{Распределение Коши}
\begin{figure}[H]
	\begin{center}
		\includegraphics[scale=0.7]{fig/Cauchy_distr.png}
		\caption{Экспериментальные и теоретические данные для распределения Коши} 
		\label{pic:pic_name} % название для ссылок внутри кода
	\end{center}
\end{figure}

\subsection{Распределение Лапласа}
\begin{figure}[H]
	\begin{center}
		\includegraphics[scale=0.7]{fig/Laplace_distr.png}
		\caption{Экспериментальные и теоретические данные для распределения Лапласа} 
		\label{pic:pic_name} % название для ссылок внутри кода
	\end{center}
\end{figure}

\subsection{Распределение Пуассона}
\begin{figure}[H]
	\begin{center}
		\includegraphics[scale=0.7]{fig/Poisson_distr.png}
		\caption{Экспериментальные и теоретические данные для распределения Пуассона} 
		\label{pic:pic_name} % название для ссылок внутри кода
	\end{center}
\end{figure}

\subsection{Равномерное распределение}


\begin{figure}[H]
	\begin{center}
		\includegraphics[scale=0.7]{fig/Uniform_distr.png}
		\caption{Экспериментальные и теоретические данные для равномерного распределения} 
		\label{pic:pic_name}
	\end{center}
\end{figure}


\section{Обсуждение}
Гистограммы для данных, сгенерированных для различных распределений, не совпадают с теоретической оценкой - графиками плотности вероятности этих распределений. В некоторых случаях данные имеют большие характеристики рассеяния (например, для некоторых массивов данных распределения Коши и равномерного распределения). Также, на некоторых графиках можно заметить смещение нормы относительно теоретической оценки (например, для распределения Лапласа во всех трёх случаях). 
\\
Важно отметить общую тенденцию - чем больше массив данных, тем лучше он соответствует (по визуальным оценкам графиков) теоретическим оценкам распределений.




\section{Литература}
Максимов Ю. Д. Математическая статистика //СПб.: СПбГПУ. – 2004.

Гурский Е. И. Теория вероятностей с элементами математической статистики. – Высш. школа, 1971.

\section{Приложения}

Репозиторий с кодом программы и кодом отчёта: \href{https://github.com/unjamini/math-stat-lab}{https://github.com/unjamini/math-stat-lab}



