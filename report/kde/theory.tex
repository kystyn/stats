\subsection{Эмперическая функция распределения}
\subsubsection{Статистический ряд}	Статистический ряд -- это упорядоченная по возрастанию последовательность \textit{различных} элементов выборки $\{z_i\}_{i = \overline{1,k}}$ с  указанием частот $\{n_i\}_{i = \overline{1,k}}$, с которыми элементы с соответствующим индексом встречаются в исходной выборке.

\begin{tabular}{|c|c|c|c|c|}
	\hline
	$z$ & $z_1$ & $z_2$ & $\ldots$ & $z_k$ \\
	\hline
	$n$ & $n_1$ & $n_2$ & $\ldots$ & $n_k$ \\
	\hline
\end{tabular}

\subsubsection{Определение}
Эмпирической функцией распределения называется функция, которая заданному вещественному числу $x$ сопоставляет относительную частоту события $X \leq x$, полученную по данной выборке:

\begin{equation}\label{edf}
F^*(x)=P^*(X \leq x)
\end{equation}

\subsubsection{Вычисление}
Для того чтобы посчитать эмпирическую функцию вероятности в заданной точке, можно построить статистический рад и просуммировать частоты, с которыми встречаются все элементы, меньшие, чем $x$  -- таким образом мы получим количество возможных событий $X \leq x $. Для вычисления относительной частоты остаётся разделить полученное значение на общее количество событий.

\begin{equation}\label{edf}
F^*(x)=P^*(X \leq x) = \displaystyle \frac{1}{n}\sum_{z_i \leq x}{n_i}
\end{equation}

где $n$ -- число событий

\subsection{Оценки плотности вероятности}
\subsubsection{Определение}

Оценкой плотности вероятности называется функция $\hat{f}(x)$, построенная по выборке и приближённо равная плотности вероятности:

\begin{equation}
\hat{f}(x) \approx f(x)
\end{equation}

\subsubsection{Ядерные оценки}

Будем оценивать плотность вероятности следующим образом:

\begin{equation}
\hat{f_n}(x) = \displaystyle \frac{1}{nh_n}\sum_{i=1}^{n}{K\left(\frac{x - x_i}{h_n}\right)}
\end{equation}

Здесь $K(u)$ -- ядро: функция, обладающая свойствами:
\begin{enumerate}
	\item \begin{equation}K \in L^1 \end{equation}
	\item \begin{equation}K(u) \geq 0 \end{equation}   
	\item \begin{equation}|K\|_{L^1}=1\end{equation} \label{kde_norm}
\end{enumerate}

Эти требования также будут выполнены и для описанной суммы: 1 -- поскольку $L^1$ -- линейное пространство, 2 -- очевидно, 3 -- по свойствам нормы. Таким образом, $\hat{f}$ удовлетворяет определению плотности вероятности.

Число $h_n$ называется шириной полосы пропускания.
Последовательность $\{h_n\}_{n \in \mathbb{N}}$ должна удовлетворять следующим условиям:
\begin{enumerate}
\item 
	\begin{equation}
	h_n \underset{n \rightarrow \infty}{\longrightarrow} 0
	\end{equation}
\item 
	\begin{equation}
	\frac{h_n}{n^{-1}} \underset{n \rightarrow \infty}{\longrightarrow} \infty
	\end{equation}
\end{enumerate}

При выполнении всех условий данные оценки называются ядерными.

Ядерная оценка является состоятельной в том смысле, что сходится по распределению к плотности вероятности случайной величины ([3, стр. 38, ф-ла 4]).

В качестве ядра будем использовать нормальное ядро:
\begin{equation}
K(u)=\frac{1}{\sqrt{2\pi}}e^{-\frac{u^2}{2}}
\end{equation}

Для вычисления ширины полосы пропускания будем использовать эмпирическое правило Сильвермана:
\begin{equation}
h_n=1.06\hat{\sigma}n^{-1/5},
\end{equation}
где $\hat{\sigma}$ -- выборочное стандартное отклонение.
