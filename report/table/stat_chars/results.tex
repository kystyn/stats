\subsection{Характеристики положения и рассеяния}
Ниже представлены таблицы полученных характеристик положения заданных распределений и характеристик рассеяния для совокупности 1000 проведённых экспериментов

\begin{table}[H]
	\begin{center}
		\begin{tabular}{|c|c|c|c|c|c|}
\hline
 & $\overline{x}$ (\ref{mean}) & $med x$ (\ref{med}) & $z_R$ (\ref{zr}) & $z_Q$ (\ref{zq}) & $z_{tr}$ (\ref{tr_mean})\\
\hline
Normal 10 &  &  &  &  & \\
\hline
$E(z)$ (\ref{ez}) & 0.0 & 0.0 & 0.0 & 0.3 & 0.0\\
\hline
$D(z)$ (\ref{dz}) & 0.1 & 0.1 & 0.2 & 0.1 & 0.1\\
\hline
Normal 100 &  &  &  &  & \\
\hline
$E(z)$ (\ref{ez}) & 0.00 & 0.00 & 0.00 & 0.04 & -0.01\\
\hline
$D(z)$ (\ref{dz}) & 0.01 & 0.02 & 0.10 & 0.01 & 0.01\\
\hline
Normal 1000 &  &  &  &  & \\
\hline
$E(z)$ (\ref{ez}) & 0.00033 & 0.001 & 0.01 & 0.005 & 0.001\\
\hline
$D(z)$ (\ref{dz}) & 0.00010 & 0.002 & 0.06 & 0.001 & 0.001\\
\hline
\end{tabular}


		\caption{Нормальное распределение}
		\label{tabl:tabl_name}
	\end{center}
\end{table}

\begin{table}[H]
	\begin{center}
		\begin{tabular}{|c|c|c|c|c|c|}
\hline
 & $\overline{x}$ (\ref{mean}) & $med x$ (\ref{med}) & $z_R$ (\ref{zr}) & $z_Q$ (\ref{zq}) & $z_{tr}$ (\ref{tr_mean})\\
\hline
Cauchy 10 &  &  &  &  & \\
\hline
$E(z)$ (\ref{ez}) & 0.0 & 0.0 & 0.2 & 1.2 & 0.0\\
\hline
$D(z)$ (\ref{dz}) & 246.8 & 0.3 & 5918.0 & 5.4 & 0.6\\
\hline
Cauchy 100 &  &  &  &  & \\
\hline
$E(z)$ (\ref{ez}) & 1.4 & 0.00 & 70.7 & 0.09 & 0.00\\
\hline
$D(z)$ (\ref{dz}) & 11369.5 & 0.03 & 28386516.2 & 0.06 & 0.03\\
\hline
Cauchy 1000 &  &  &  &  & \\
\hline
$E(z)$ (\ref{ez}) & -0.7 & 0.001 & -354.9 & 0.010 & 0.001\\
\hline
$D(z)$ (\ref{dz}) & 442.9 & 0.002 & 108911495.3 & 0.005 & 0.003\\
\hline
\end{tabular}


		\caption{Распределение Коши}
		\label{tabl:tabl_name}
	\end{center}
\end{table}

\begin{table}[H]
	\begin{center}
		\begin{tabular}{|c|c|c|c|c|c|}
\hline
 & $\overline{x}$ (\ref{mean}) & $med x$ (\ref{med}) & $z_R$ (\ref{zr}) & $z_Q$ (\ref{zq}) & $z_{tr}$ (\ref{tr_mean})\\
\hline
Laplace 10 &  &  &  &  & \\
\hline
$E(z)$ (\ref{ez}) & -0.02 & 0.00 & 0.0 & 0.3 & -0.01\\
\hline
$D(z)$ (\ref{dz}) & 0.09 & 0.07 & 0.4 & 0.1 & 0.07\\
\hline
Laplace 100 &  &  &  &  & \\
\hline
$E(z)$ (\ref{ez}) & -0.004 & -0.001 & 0.0 & 0.038 & -0.003\\
\hline
$D(z)$ (\ref{dz}) & 0.010 & 0.006 & 0.4 & 0.009 & 0.006\\
\hline
Laplace 1000 &  &  &  &  & \\
\hline
$E(z)$ (\ref{ez}) & -0.0004 & -0.0002 & 0.0 & 0.000 & -0.0003\\
\hline
$D(z)$ (\ref{dz}) & 0.0010 & 0.0005 & 0.4 & 0.001 & 0.0006\\
\hline
\end{tabular}


		\caption{Распределение Лапласа}
		\label{tabl:tabl_name}
	\end{center}
\end{table}

\begin{table}[H]
	\begin{center}
		\begin{tabular}{|c|c|c|c|c|c|}
\hline
 & $\overline{x}$ (\ref{mean}) & $med x$ (\ref{med}) & $z_R$ (\ref{zr}) & $z_Q$ (\ref{zq}) & $z_{tr}$ (\ref{tr_mean})\\
\hline
Poisson 10 &  &  &  &  & \\
\hline
$E(z)$ (\ref{ez}) & 10.0 & 9.9 & 10.3 & 10.9 & 9.89\\
\hline
$D(z)$ (\ref{dz}) & 1.0 & 1.4 & 2.0 & 1.4 & 1.09\\
\hline
Poisson 100 &  &  &  &  & \\
\hline
$E(z)$ (\ref{ez}) & 10.0 & 9.8 & 10.9 & 10.0 & 9.8\\
\hline
$D(z)$ (\ref{dz}) & 0.1 & 0.2 & 1.0 & 0.2 & 0.1\\
\hline
Poisson 1000 &  &  &  &  & \\
\hline
$E(z)$ (\ref{ez}) & 10.00 & 9.993 & 11.7 & 9.993 & 9.85\\
\hline
$D(z)$ (\ref{dz}) & 0.01 & 0.007 & 0.7 & 0.004 & 0.01\\
\hline
\end{tabular}


		\caption{Распределение Пуассона}
		\label{tabl:tabl_name}
	\end{center}
\end{table}

\begin{table}[H]
	\begin{center}
		\begin{tabular}{|c|c|c|c|c|c|}
\hline
 & $\overline{x}$ (\ref{mean}) & $med x$ (\ref{med}) & $z_R$ (\ref{zr}) & $z_Q$ (\ref{zq}) & $z_{tr}$ (\ref{tr_mean})\\
\hline
Uniform 10 &  &  &  &  & \\
\hline
$E(z)$ (\ref{ez}) & 0.0 & 0.0 & 0.01 & 0.3 & 0.0\\
\hline
$D(z)$ (\ref{dz}) & 0.1 & 0.2 & 0.05 & 0.1 & 0.2\\
\hline
Uniform 100 &  &  &  &  & \\
\hline
$E(z)$ (\ref{ez}) & 0.00 & 0.00 & 0.0007 & 0.05 & 0.00\\
\hline
$D(z)$ (\ref{dz}) & 0.01 & 0.03 & 0.0005 & 0.02 & 0.02\\
\hline
Uniform 1000 &  &  &  &  & \\
\hline
$E(z)$ (\ref{ez}) & 0.0003 & 0.0005 & -0.000161 & 0.005 & 0.000\\
\hline
$D(z)$ (\ref{dz}) & 0.0010 & 0.0028 & 0.000006 & 0.001 & 0.002\\
\hline
\end{tabular}


		\caption{Равномерное распределение}
		\label{tabl:tabl_name}
	\end{center}
\end{table}
