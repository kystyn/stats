\subsection{Выборочные числовые характеристики}
Для того, чтобы анализировать получаемые данные, вводятся характеристики положения -- такие функции, которые различными способами усредняют данные и демонстрируют общие закономерности, и характеристики рассеяния -- это характеристики, которые показывают насколько данные разбросаны относительно своих характеристик положения.

\subsection{Характеристики положения}
\begin{itemize}
	\item Выборочное среднее
	
	\begin{equation}\label{mean}
		\overline{x}=\displaystyle \sum_{i=1}^{n} {x_i}
	\end{equation}
	
	\item Выборочная медиана
	
	\begin{equation}\label{med}
		med x =
		\begin{cases}
		x_{(l+1)}, & n=2l + 1 \\
		\frac{x_{(l)}+x_{(l+1)}}{2}, & n=2l
		\end{cases}
	\end{equation}

	\item Полусумма экстремальных выборочных элементов

	\begin{equation}\label{zr}	
		z_R =\frac{x_{(1)}+x_{(n)}}{2}
	\end{equation}
	
	\item Полусумма квартилей
	
	Выборочная квартиль порядка $p$ определяется как:
	
	$$z_p =
	\begin{cases}
	x_{([np]+1)}, & np \in \mathbb{Q} \backslash \mathbb{Z} \\
	x_{(np)}, & np \in \mathbb{Z}
	\end{cases}
	$$
	
	\begin{equation}\label{zq}
		z_Q =\frac{z_{1/4}+z_{3/4}}{2}
	\end{equation}
	
	\item Усечённое среднее
	
	\begin{equation}\label{tr_mean}
		z_{tr}=\frac{1}{n-2r}\displaystyle \sum_{i=r+1}^{n-r} x_{(i)}, r \approx \frac{n}{4}
	\end{equation}
\end{itemize}

\subsection{Характеристики рассеяния}
\begin{itemize}
	\item Дисперсия
	
	\begin{equation}\label{disp}
		D(X)=\frac{1}{n}\displaystyle \sum_{i=1}^{n}(x_i-\overline{x})^2
	\end{equation}
	
\end{itemize}
