\subsection{Ядерные оценки плотности}

Эмпирическая функция распределения на всех распределениях достаточно точно приближается к истинной функции распределения, причём это свойство сохраняется даже на выборке из двадцати элементов: оценивая на глаз, можно сказать, что точечные отличия для непрерывных распределений не превосходят 10\%, а для распределения Пуассона достигли 20\% на выборке из 
 элементов.
 
 Более интересно дела обстоят с ядерной оценкой плотности: эмпирическая ширина полосы пропускания Сильвермана оказалась оптимальной не везде: так, у распределения Лапласа при всех тестируемых размерах выборок ядерная оценка получилась куда точнее, когда полоса пропускания была в два раза уже, чем оптимальная
 
 В целом ядерная оценка получилась наиболее гладкой в смысле количества экстремальных точек при интервале пропускания, чья ширина в два раза больше оптимального. То есть оптимальная оценка иногда, быть может, обеспечивает худшую поточечную сходимость, но лучшую интегральную ($L^p$).
 
 Стоит отметить, что ядерная оценка дала очень негладкое приближение распределения Пуассона при половинном от оптимального интервале пропускания. При оставшихся интервалах пропускания результаты получились более удовлетворительными в смысле гладкости, но всё ещё очень неточными в смысле интегральной метрики до истинной плотности распределения.
 Вероятно, это обосновано принципиально дискретным характером распределения, что осложняет сходимость на малых размерах выборки.
