\item Для каждого распределения требуется сгенерировать выборки из 20, 60 и 100 элементов.

Построить на них эмпирические функции распределения и ядерные оценки плотности распределения:
\begin{itemize}
	\item Для непрерывных распределений -- на отрезке $[-4; 4]$
	\item Для распределения Пуассона -- на отрезке $[6; 14]$
\end{itemize} .

Для каждого распределения требуется:
\begin{itemize}
	\item Определить долю выбросов, сгенерировав выборку, соответствующую распределению, 1000 раз и вычислив среднюю долю выбросов
	\item Сравнить с результатами, полученными теоретически
\end{itemize}

\end{enumerate}
