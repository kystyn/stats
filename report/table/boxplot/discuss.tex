\subsection{БоксплотТьюки}

Боксплоты Тьюки действительно позволяют более наглядно и с меньшими усилиями оценивать важные характеристики распределений. Так, исходя из полученных рисунков, наглядно видно то, что мы довольно трудоёмко анализировали в предыдущих частях: чем больше количество элементов в сгенерированной выборке, тем сильнее медиана и интерквартальный размах приближаются к теоретическим значениям. Также хорошо видно, что нормальное распределение является в высшей степени симметричным, даже на выборке в 20 значений. Мы видим, что распределение Коши имеет много выбросов, находящихся на значительном удалении (расстояние от медианы до выброса в десятки раз выше интерквартального размаха) от медианы -- в этом смысле данное распределение предпочительно использовать при генерации случайных чисел.

Стоит также отметить, что у всех распределений медиана смещена относительно середины межквартильного интервала. 

Вспомним предыдущие результаты: ранее было получено, что медиана среди прочих характеристик максимально приближена к выборочному среднему, которое в свою очередь сходится почти наверное к матожиданию (при его наличии) [2]. То есть медиана является очень состоятельной оценкой теоретического среднего. Этот факт позволяет нам лишний раз убедиться, что полусумма квартилей не является достаточно качественной оценкой среднего.

Что касается теоретической и экспериментальной доли выбросов: здесь мы видим, что экспериментальная доля выбросов оказывается тем точнее приближена к теоретической вероятности выбросов, чем больше элементов в выборке. В таблице снова видно, что доля выбросов в распределении Коши значительно превосходит долю выбросов в других распределениях.
