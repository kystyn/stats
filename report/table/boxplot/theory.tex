\subsection{Боксплот Тьюки}
Самым простым средством, использующимся для визуального анализа статистических данных, является гистограмма. Однако она не всегда является удобной: по общему виду диаграммы зачастую бывает трудно понять, какими параметрами положения и рассеяния обладает исследуемое распределение.

В связи с этим было разработано такое средство визуализации распределения, как \textit{боксплот Тьюки} -- это прямоугольник, вертикальные границы которого соответствуют нижнему и верхнему квартилям. Внутри него отображается вертикальная линия, соответствующая медиане. Также из этого ``ящика'' выходят ``усы'', сонаправленные абсциссе. Их концы соответствуют точкам, отстоящим от соответствующих квартилей на полтора межквартильных расстояния:
\begin{equation}
	X_1 = Q_1 - \frac{3}{2}(Q_3-Q_1)
\end{equation}

\begin{equation}
	X_2 = Q_3 + \frac{3}{2}(Q_3-Q_1)
\end{equation}

Они характеризуют меру рассеяния данных. Все данные, которые оказались за пределами усов, можно считать статистическими выбросами. Их отображают с помощью кружочков. Таким образом, мы можем легко оценить степень разброса и асимметрии данных.

По ординате боксплот можно располагать, например, таким образом, чтобы его средняя линия соответствовала количеству элементов, из которых составлена данная выборка. Можно также располагать несколько боксплотов друг над другом, чтобы визуально сравнивать полученные параметры распределений.

\begin{figure}[H]
	\begin{center}
		\includegraphics[scale=0.3]{boxplot/example_boxplot}
		\caption{Пример боксплота (\ref{bplotex})}
		\label{pic:pic_name}	
	\end{center}
\end{figure}

\subsection{Вычисление вероятности выбросов}

Выбросами считаются величины, удовлетворяющие условию:


\begin{equation}
\left[
\begin{array}{c}
x < X_1 \\
x > X_2
\end{array}
\right.
\end{equation}

Для теоретических распределений эту величину можно получить аналитически:

\begin{equation}\label{trashdata}
P_{\text{в}}^{\text{Т}}=P(x<X_1 | x > X_2) = F_X(X_1) + 1 - F_X(X_2)
\end{equation}

где $F_X$ -- функция распределения.

Для экспериментальных -- непосредственным вычислением: итерируясь по всем элементам сгенерированного распределения, увеличивать счётчик выбросов в соответствующем случае. В конце поделить полученное количество выбросов на общее число элементов в выборке.