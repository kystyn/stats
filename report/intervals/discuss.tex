\subsection{Доверительные интервалы для матожидания и дисперсии}

Для обоих подходов характерно, что при увеличении размера выборки доверительный интервал сужается, что является логичным, учитывая характер методов, с помощью которых были получены оценки.

Также видно, что асимптотический подход даёт значительно более широкий доверительный интервал для дисперсии на выборке из 20 элементов, что обусловлено собственно природой подхода: мы производим асимптотические оценки там, где они таковыми не являются.

При этом результаты для матожидания получились похожими в обоих подходах, потому что идея оценок в подходах абсолютно идентична. Отличаются лишь только функции распределения, фигурирующие в соответствующей оценке. При этом вид этих функций очень похож для любых распределений.

Таким образом, асимптотический подход можно рекомендовать для оценки матожидания любого распределения даже на сравнительно небольших выборках, чего, однако, нельзя сказать про дисперсию.

На выборке из 100 элементов методы дали схожие результаты, из чего можно сделать вывод, что асимптотический подход пригоден для применения на выборках такого размера.
