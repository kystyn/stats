\section{Доверительные оценки для параметров нормального распределения}

\subsection{Доверительный интервал для матожидания $m$ нормального распределения}

Для выборки $(x_1, ..., x_n)$ из нормальной генеральной совокупности найдём среднее $\overline{x}$ и среднее квадратичное отклонение $s$.

Тогда величина 

\begin{equation}
	T = \sqrt{n - 1} \cdot \frac{\overline{x} - m}{s},
\end{equation}

называемая статистикой Стьюдента, распределена по закону Стьюдента с $n-1$ степенями свободы.

Произведя несложные преобразования, получим, что:

\begin{equation}
P \left( -x < T < x \right) = 2F_T(x) - 1,
\end{equation}

где $F_T$ -- функция распределения Стьюдента с $n-1$ степенями свободы.

Полагая $2F_T(x) - 1 = 1 - \alpha$, где $\alpha$ -- уровень значимости, имеем:

\begin{multline}
\displaystyle P \left( \overline{x} - \frac{sx}{\sqrt{n-1}} < m < \overline{x} + \frac{sx}{\sqrt{n-1}} \right) = \\
= P \left( \overline{x} - \frac{st_{1 - \frac{\alpha}{2}}(n - 1)}{\sqrt{n-1}} < m < \overline{x} + \frac{st_{1 - \frac{\alpha}{2}}(n - 1)}{\sqrt{n-1}} \right) = 1 - \alpha
\end{multline}

И таким образом получаем доверительный интервал для матожидания с вероятностью $1 - \alpha$.

\subsection{Доверительный интервал для среднего квадратического отклонения $\sigma$ нормального распределения}

Доказано, что случайная величина $\frac{ns^2}{\sigma^2}$ распределена по закону $\chi^2$ с $n-1$ степенями свободы.

После ряда преобразований, получаем:

\begin{equation}
	\displaystyle P \left( \frac{s \sqrt{n}} {\sqrt{\chi_{1 - \alpha/2}^2(n-1)}} < \sigma <  \frac{s \sqrt{n}} {\sqrt{\chi_{\alpha/2}^2(n-1)}} \right)
\end{equation}

\section{Доверительные оценки для параметров произвольного распределения. Асимптотический подход}

\subsection{Доверительные оценки для матожидания при большом размере выборки}

Если исследуемое распределение имеет конечное матожидание и дисперсию, то имеет место центральная предельная теорема:

\begin{equation}
\frac{\overline{x}-\mathbf{M}x}{\sqrt{\mathbf{Dx}}}=\sqrt{n} \cdot \frac{\overline{x} - m}{\sigma} \overset{F}{\longrightarrow} N(x, 0, 1)
\end{equation}

Отсюда получаем, что

\begin{equation*}
P \left(-x < \sqrt{n} \cdot \frac{\overline{x} - m}{\sigma} < x \right) \approx 2 \Phi(x),
\end{equation*}

где $\Phi(x)$ -- функция Лапласа.

Полагая $u_{1 - \alpha / 2}$ за соответствующий квантиль центрированного нормального распределения с единичной дисперсией, получаем:

\begin{equation}
P \left(\overline{x} - \frac{su_{1 - \alpha / 2}}{\sqrt{n}} < m < \overline{x} + \frac{su_{1 - \alpha / 2}}{\sqrt{n}} \right) \approx \gamma,
\end{equation}

что и даёт доверительный интервал для матожидания $m$ с доверительной вероятностью $\gamma$.

\subsection{Доверительные оценки для дисперсии при большом размере выборки}

Используя ЦПТ и разложение в ряд Тейлора для характеристической функции Лапласа, получим, что:

\begin{equation}
s(1 + U)^{-1/2} < \sigma < s(1 - U)^{-1/2},
\end{equation}

где $U = u_{1 - \alpha / 2} \sqrt{\frac{e + 2}{n}}$