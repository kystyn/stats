\section{Постановка задачи}
Даны пять распределений:
\begin{itemize}
\item Нормальное распределение: $N(x, 0, 1)$
\item Распределение Коши: $C(x, 0, 1)$
\item Распределение Лапласа: $L(x, 0, \frac{1}{\sqrt{2}})$
\item Распределение Пуассона: $P(k, 10)$
\item Равномерное распределение: $U(x, -\sqrt{3}, \sqrt{3})$
\end{itemize}

Для каждого распределения требуется сгенерировать выборки из 10, 100 и 1000 элементов. 

Требуется вычислить статистические характеристики положения:
\begin{itemize}
	\item $\overline{x}$
	\item $med x$
	\item $z_R, z_Q, z_{tr}$
\end{itemize}

Такие вычисления необходимо повторить 1000 раз и найти среднее и дисперсию уже для этих характеристик:

\begin{equation}\label{ez}
	E(z)=\overline{z}
\end{equation}

\begin{equation}\label{dz}
	D(z)=\overline{z^2}-\overline{z}^2
\end{equation}

\section{Теория}

Для того, чтобы анализировать получаемые данные, вводятся характеристики положения -- такие функции, которые различными способами усредняют данные и демонстрируют общие закономерности, и характеристики рассеяния -- это характеристики, которые показывают насколько данные разбросаны относительно своих характеристик положения.

\subsection{Характеристики положения}
\begin{itemize}
	\item Выборочное среднее
	
	\begin{equation}\label{mean}
		\overline{x}=\displaystyle \sum_{i=1}^{n} {x_i}
	\end{equation}
	
	\item Выборочная медиана
	
	\begin{equation}\label{med}
		med x =
		\begin{cases}
		x_{(l+1)}, & n=2l + 1 \\
		\frac{x_{(l)}+x_{(l+1)}}{2}, & n=2l
		\end{cases}
	\end{equation}

	\item Полусумма экстремальных выборочных элементов

	\begin{equation}\label{zr}	
		z_R =\frac{x_{(1)}+x_{(n)}}{2}
	\end{equation}
	
	\item Полусумма квартилей
	
	Выборочная квартиль порядка $p$ определяется как:
	
	$$z_p =
	\begin{cases}
	x_{([np]+1)}, & np \in \mathbb{Q} \backslash \mathbb{Z} \\
	x_{(np)}, & np \in \mathbb{Z}
	\end{cases}
	$$
	
	\begin{equation}\label{zq}
		z_Q =\frac{z_{1/4}+z_{3/4}}{2}
	\end{equation}
	
	\item Усечённое среднее
	
	\begin{equation}\label{tr_mean}
		z_{tr}=\frac{1}{n-2r}\displaystyle \sum_{i=r+1}^{n-r} x_{(i)}, r \approx \frac{n}{4}
	\end{equation}
\end{itemize}

\subsection{Характеристики рассеяния}
\begin{itemize}
	\item Дисперсия
	
	\begin{equation}\label{disp}
		D(X)=\frac{1}{n}\displaystyle \sum_{i=1}^{n}(x_i-\overline{x})^2
	\end{equation}
	
\end{itemize}
\section{Реализация}
Данная работа реализована на языке программирования Python с использованием IDE PyCharm и библиотеки NumPy в ОС Ubuntu 19.04.

Отчёт подготовлен с помощью компилятора pdflatex и среды разработки TeXStudio.

\section{Результаты}
Ниже представлены таблицы полученных характеристик положения заданных распределений и характеристик рассеяния для совокупности 1000 проведённых экспериментов

\begin{table}[H]
	\begin{center}
		\input{table/normal.tex}
		\caption{Нормальное распределение}
		\label{tabl:tabl_name}
	\end{center}
\end{table}

\begin{table}[H]
	\begin{center}
		\input{table/cauchy.tex}
		\caption{Распределение Коши}
		\label{tabl:tabl_name}
	\end{center}
\end{table}

\begin{table}[H]
	\begin{center}
		\input{table/laplace.tex}
		\caption{Распределение Лапласа}
		\label{tabl:tabl_name}
	\end{center}
\end{table}

\begin{table}[H]
	\begin{center}
		\input{table/poisson.tex}
		\caption{Распределение Пуассона}
		\label{tabl:tabl_name}
	\end{center}
\end{table}

\begin{table}[H]
	\begin{center}
		\input{table/uniform.tex}
		\caption{Равномерное распределение}
		\label{tabl:tabl_name}
	\end{center}
\end{table}


\section{Обсуждение}
Гистограммы в целом визуально повторяют графики плотности распределения, причём чем выше количество случайных величин, тем выше сходство теоретической функции плотности распределения и кривой, огибающей верхние точки столбцов гистограммы (суть, экспериментальной функции плотности распределения).

Однако существуют определённого рода различия: так, максимумы гистограмм и плотностей распределения почти нигде не совпали. Более того, соотношение, кто из них больше, может меняться от эксперимента к эксперименту (распределение Лапласа). Также наблюдаются всплески гистограмм, что наиболее хорошо прослеживается на распределении Коши.



\section{Литература}
Максимов Ю. Д. Математическая статистика //СПб.: СПбГПУ. – 2004.

\section{Приложения}

Репозиторий с кодом программы и кодом отчёта: \href{https://github.com/kystyn/stats}{https://github.com/kystyn/stats}



