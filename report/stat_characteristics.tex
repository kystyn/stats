\section{Постановка задачи}
Требуется сгенерировать наборы случайных точек для пяти распределений:
\begin{itemize}
\item Нормальное распределение: $N(x, 0, 1)$
\item Распределение Коши: $C(x, 0, 1)$
\item Распределение Лапласа: $L(x, 0, \frac{1}{\sqrt{2}})$
\item Распределение Пуассона: $P(k, 10)$
\item Равномерное распределение: $U(x, -\sqrt{3}, \sqrt{3})$
\end{itemize}

Для каждого распределения сгенерировать выборки из 10, 100 и 1000 элементов. 

Требуется вычислить статистические характеристики положения:
\begin{itemize}
	\item $\overline{x}$
	\item $med x$
	\item $z_R, z_Q, z_{tr}$
\end{itemize}

Такие вычисления необходимо повторить 1000 раз и найти среднюю и дисперсию уже для этих характеристик:

$E(z)=\overline{z}$

$D(z)=\overline{z^2}-\overline{z}^2$

\section{Теория}

Для того, чтобы анализировать получаемые данные, вводятся характеристики положения -- такие функции, которые различными способами усредняют данные и демонстрируют общие закономерности, и характеристики рассеяния -- это характеристики, которые показывают насколько данные разбросаны относительно своих характеристик положения.

\subsection{Характеристики положения}
\begin{itemize}
	\item Выборочное среднее
	
	$\overline{x}=\displaystyle \sum_{i=1}^{n} {x_i}$
	
	\item Выборочная медиана
	
	$med x =
	\begin{cases}
		x_{l+1}, & n=2l + 1 \\
		\frac{x_{l}+x_{l+1}}{2}, & n=2l
	\end{cases}
	$
	
	\item Полусумма экстремальных выборочных элементов
	
	$z_R =\frac{x_{1}+x_{n}}{2}$
	
	\item Полусумма квартилей
	
	Выборочная квартиль порядка $p$ определяется как:
	$z_p =
	\begin{cases}
	x_{[np]+1}, & np \in \mathbb{Q} \backslash \mathbb{Z} \\
	x_{np}, & np \in \mathbb{Z}
	\end{cases}
	$
	
	$z_Q =\frac{z_{1/4}+z_{3/4}}{2}$
	
	\item Усечённое среднее
	
	$z_{tr}=\frac{1}{n-2r}\displaystyle \sum_{i=1}^{n-r} x_{i}, r \approx \frac{n}{4}$
\end{itemize}

\subsection{Характеристики рассеяния}
\begin{itemize}
	\item Дисперсия
	
	$D(X)=\frac{1}{n}\displaystyle \sum_{i=1}^{n}(x_i-\overline{x})^2$
	
\end{itemize}
\section{Реализация}
Данная работа реализована на языке программирования Python с использованием IDE PyCharm и библиотеки NumPy в ОС Ubuntu 19.04.

Отчёт подготовлен с помощью компилятора pdflatex и среды разработки TeXStudio.

\section{Результаты}
На нижележащих рисунках изображены гистограммы распределений и теоретические функции плотности распределения

\subsection{Нормальное распределение}
\begin{figure}[H]
	\begin{center}
		\includegraphics[scale=0.5]{norm}
		\caption{Гистограмма и плотность вероятности для нормального распределения} 
		\label{pic:pic_name}
	\end{center}
\end{figure}

\subsection{Распределение Коши}
\begin{figure}[H]
	\begin{center}
		\includegraphics[scale=0.5]{cauchy}
		\caption{Гистограмма и плотность вероятности для распределения Коши} 
		\label{pic:pic_name} 
	\end{center}
\end{figure}

\subsection{Распределение Лапласа}
\begin{figure}[H]
	\begin{center}
		\includegraphics[scale=0.5]{laplace}
		\caption{Гистограмма и плотность вероятности для распределения Лапласа} 
		\label{pic:pic_name} 
	\end{center}
\end{figure}

\subsection{Распределение Пуассона}
\begin{figure}[H]
	\begin{center}
		\includegraphics[scale=0.5]{poisson}
		\caption{Гистограмма и плотность вероятности для распределения Пуассона} 
		\label{pic:pic_name} 
	\end{center}
\end{figure}

\subsection{Равномерное распределение}
\begin{figure}[H]
	\begin{center}
		\includegraphics[scale=0.5]{uniform}
		\caption{Гистограмма и плотность вероятности для равномерного распределения} 
		\label{pic:pic_name}
	\end{center}
\end{figure}


\section{Обсуждение}
Гистограммы в целом визуально повторяют графики плотности распределения, причём чем выше количество случайных величин, тем выше сходство теоретической функции плотности распределения и кривой, огибающей верхние точки столбцов гистограммы (суть, экспериментальной функции плотности распределения).

Однако существуют определённого рода различия: так, максимумы гистограмм и плотностей распределения почти нигде не совпали. Более того, соотношение, кто из них больше, может меняться от эксперимента к эксперименту (распределение Лапласа). Также наблюдаются всплески гистограмм, что наиболее хорошо прослеживается на распределении Коши.



\section{Литература}
Максимов Ю. Д. Математическая статистика //СПб.: СПбГПУ. – 2004.

\section{Приложения}

Репозиторий с кодом программы и кодом отчёта: \href{https://github.com/kystyn/stats}{https://github.com/kystyn/stats}



