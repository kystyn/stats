\section{Постановка задачи}
Даны пять распределений:
\begin{itemize}
\item Нормальное распределение: $N(x, 0, 1)$
\item Распределение Коши: $C(x, 0, 1)$
\item Распределение Лапласа: $L(x, 0, \frac{1}{\sqrt{2}})$
\item Распределение Пуассона: $P(k, 10)$
\item Равномерное распределение: $U(x, -\sqrt{3}, \sqrt{3})$
\end{itemize}

Для каждого распределения требуется сгенерировать выборки из 20 и 100 элементов.

Построить для них боксплот Тьюки.

Для каждого распределения требуется определить долю выбросов, сгенерировав выборку, соответствующую распределению, 1000 раз и вычислив среднюю долю выбросов, и сравнить с результатами, полученными теоретически.

\section{Теория}

Для того, чтобы анализировать получаемые данные, вводятся характеристики положения -- такие функции, которые различными способами усредняют данные и демонстрируют общие закономерности, и характеристики рассеяния -- это характеристики, которые показывают насколько данные разбросаны относительно своих характеристик положения.

\subsection{Характеристики положения}
\begin{itemize}
	\item Выборочное среднее
	
	\begin{equation}\label{mean}
		\overline{x}=\displaystyle \sum_{i=1}^{n} {x_i}
	\end{equation}
	
	\item Выборочная медиана
	
	\begin{equation}\label{med}
		med x =
		\begin{cases}
		x_{(l+1)}, & n=2l + 1 \\
		\frac{x_{(l)}+x_{(l+1)}}{2}, & n=2l
		\end{cases}
	\end{equation}

	\item Полусумма экстремальных выборочных элементов

	\begin{equation}\label{zr}	
		z_R =\frac{x_{(1)}+x_{(n)}}{2}
	\end{equation}
	
	\item Полусумма квартилей
	
	Выборочная квартиль порядка $p$ определяется как:
	
	$$z_p =
	\begin{cases}
	x_{([np]+1)}, & np \in \mathbb{Q} \backslash \mathbb{Z} \\
	x_{(np)}, & np \in \mathbb{Z}
	\end{cases}
	$$
	
	\begin{equation}\label{zq}
		z_Q =\frac{z_{1/4}+z_{3/4}}{2}
	\end{equation}
	
	\item Усечённое среднее
	
	\begin{equation}\label{tr_mean}
		z_{tr}=\frac{1}{n-2r}\displaystyle \sum_{i=r+1}^{n-r} x_{(i)}, r \approx \frac{n}{4}
	\end{equation}
\end{itemize}

\subsection{Характеристики рассеяния}
\begin{itemize}
	\item Дисперсия
	
	\begin{equation}\label{disp}
		D(X)=\frac{1}{n}\displaystyle \sum_{i=1}^{n}(x_i-\overline{x})^2
	\end{equation}
	
\end{itemize}
\section{Реализация}
Данная работа реализована на языке программирования Python с использованием IDE PyCharm и библиотеки NumPy в ОС Ubuntu 19.04.

Отчёт подготовлен с помощью компилятора pdflatex и среды разработки TeXStudio.

\section{Результаты}
Ниже представлены таблицы полученных характеристик положения заданных распределений и характеристик рассеяния для совокупности 1000 проведённых экспериментов

\begin{table}[H]
	\begin{center}
		\begin{tabular}{|c|c|c|c|c|c|}
\hline
 & $\overline{x}$ (\ref{mean}) & $med x$ (\ref{med}) & $z_R$ (\ref{zr}) & $z_Q$ (\ref{zq}) & $z_{tr}$ (\ref{tr_mean})\\
\hline
Normal 10 &  &  &  &  & \\
\hline
$E(z)$ (\ref{ez}) & 0.0 & 0.0 & 0.0 & 0.3 & 0.0\\
\hline
$D(z)$ (\ref{dz}) & 0.1 & 0.1 & 0.2 & 0.1 & 0.1\\
\hline
Normal 100 &  &  &  &  & \\
\hline
$E(z)$ (\ref{ez}) & 0.00 & 0.00 & 0.00 & 0.04 & -0.01\\
\hline
$D(z)$ (\ref{dz}) & 0.01 & 0.02 & 0.10 & 0.01 & 0.01\\
\hline
Normal 1000 &  &  &  &  & \\
\hline
$E(z)$ (\ref{ez}) & 0.00033 & 0.001 & 0.01 & 0.005 & 0.001\\
\hline
$D(z)$ (\ref{dz}) & 0.00010 & 0.002 & 0.06 & 0.001 & 0.001\\
\hline
\end{tabular}


		\caption{Нормальное распределение}
		\label{tabl:tabl_name}
	\end{center}
\end{table}

\begin{table}[H]
	\begin{center}
		\begin{tabular}{|c|c|c|c|c|c|}
\hline
 & $\overline{x}$ (\ref{mean}) & $med x$ (\ref{med}) & $z_R$ (\ref{zr}) & $z_Q$ (\ref{zq}) & $z_{tr}$ (\ref{tr_mean})\\
\hline
Cauchy 10 &  &  &  &  & \\
\hline
$E(z)$ (\ref{ez}) & 0.0 & 0.0 & 0.2 & 1.2 & 0.0\\
\hline
$D(z)$ (\ref{dz}) & 246.8 & 0.3 & 5918.0 & 5.4 & 0.6\\
\hline
Cauchy 100 &  &  &  &  & \\
\hline
$E(z)$ (\ref{ez}) & 1.4 & 0.00 & 70.7 & 0.09 & 0.00\\
\hline
$D(z)$ (\ref{dz}) & 11369.5 & 0.03 & 28386516.2 & 0.06 & 0.03\\
\hline
Cauchy 1000 &  &  &  &  & \\
\hline
$E(z)$ (\ref{ez}) & -0.7 & 0.001 & -354.9 & 0.010 & 0.001\\
\hline
$D(z)$ (\ref{dz}) & 442.9 & 0.002 & 108911495.3 & 0.005 & 0.003\\
\hline
\end{tabular}


		\caption{Распределение Коши}
		\label{tabl:tabl_name}
	\end{center}
\end{table}

\begin{table}[H]
	\begin{center}
		\begin{tabular}{|c|c|c|c|c|c|}
\hline
 & $\overline{x}$ (\ref{mean}) & $med x$ (\ref{med}) & $z_R$ (\ref{zr}) & $z_Q$ (\ref{zq}) & $z_{tr}$ (\ref{tr_mean})\\
\hline
Laplace 10 &  &  &  &  & \\
\hline
$E(z)$ (\ref{ez}) & -0.02 & 0.00 & 0.0 & 0.3 & -0.01\\
\hline
$D(z)$ (\ref{dz}) & 0.09 & 0.07 & 0.4 & 0.1 & 0.07\\
\hline
Laplace 100 &  &  &  &  & \\
\hline
$E(z)$ (\ref{ez}) & -0.004 & -0.001 & 0.0 & 0.038 & -0.003\\
\hline
$D(z)$ (\ref{dz}) & 0.010 & 0.006 & 0.4 & 0.009 & 0.006\\
\hline
Laplace 1000 &  &  &  &  & \\
\hline
$E(z)$ (\ref{ez}) & -0.0004 & -0.0002 & 0.0 & 0.000 & -0.0003\\
\hline
$D(z)$ (\ref{dz}) & 0.0010 & 0.0005 & 0.4 & 0.001 & 0.0006\\
\hline
\end{tabular}


		\caption{Распределение Лапласа}
		\label{tabl:tabl_name}
	\end{center}
\end{table}

\begin{table}[H]
	\begin{center}
		\begin{tabular}{|c|c|c|c|c|c|}
\hline
 & $\overline{x}$ (\ref{mean}) & $med x$ (\ref{med}) & $z_R$ (\ref{zr}) & $z_Q$ (\ref{zq}) & $z_{tr}$ (\ref{tr_mean})\\
\hline
Poisson 10 &  &  &  &  & \\
\hline
$E(z)$ (\ref{ez}) & 10.0 & 9.9 & 10.3 & 10.9 & 9.89\\
\hline
$D(z)$ (\ref{dz}) & 1.0 & 1.4 & 2.0 & 1.4 & 1.09\\
\hline
Poisson 100 &  &  &  &  & \\
\hline
$E(z)$ (\ref{ez}) & 10.0 & 9.8 & 10.9 & 10.0 & 9.8\\
\hline
$D(z)$ (\ref{dz}) & 0.1 & 0.2 & 1.0 & 0.2 & 0.1\\
\hline
Poisson 1000 &  &  &  &  & \\
\hline
$E(z)$ (\ref{ez}) & 10.00 & 9.993 & 11.7 & 9.993 & 9.85\\
\hline
$D(z)$ (\ref{dz}) & 0.01 & 0.007 & 0.7 & 0.004 & 0.01\\
\hline
\end{tabular}


		\caption{Распределение Пуассона}
		\label{tabl:tabl_name}
	\end{center}
\end{table}

\begin{table}[H]
	\begin{center}
		\begin{tabular}{|c|c|c|c|c|c|}
\hline
 & $\overline{x}$ (\ref{mean}) & $med x$ (\ref{med}) & $z_R$ (\ref{zr}) & $z_Q$ (\ref{zq}) & $z_{tr}$ (\ref{tr_mean})\\
\hline
Uniform 10 &  &  &  &  & \\
\hline
$E(z)$ (\ref{ez}) & 0.0 & 0.0 & 0.01 & 0.3 & 0.0\\
\hline
$D(z)$ (\ref{dz}) & 0.1 & 0.2 & 0.05 & 0.1 & 0.2\\
\hline
Uniform 100 &  &  &  &  & \\
\hline
$E(z)$ (\ref{ez}) & 0.00 & 0.00 & 0.0007 & 0.05 & 0.00\\
\hline
$D(z)$ (\ref{dz}) & 0.01 & 0.03 & 0.0005 & 0.02 & 0.02\\
\hline
Uniform 1000 &  &  &  &  & \\
\hline
$E(z)$ (\ref{ez}) & 0.0003 & 0.0005 & -0.000161 & 0.005 & 0.000\\
\hline
$D(z)$ (\ref{dz}) & 0.0010 & 0.0028 & 0.000006 & 0.001 & 0.002\\
\hline
\end{tabular}


		\caption{Равномерное распределение}
		\label{tabl:tabl_name}
	\end{center}
\end{table}


\section{Обсуждение}

Во всех распределениях, кроме распределения Коши, можно отметить, что выборочное среднее и выборочная медиана располагаются друг другу ближе (в смысле модуля разности), чем к другим характеристикам. Также видно, что эти параметры положения являются наиболее стабильными: во всех распределениях, кроме распределения Коши, они обладают минимальной дисперсией по сравнению с другими характеристиками. Более того, вычисленные в разных экспериментах значения медианы распределения Коши также оказывались минимально разбросанными, из чего можно сделать вывод, что данную характеристику можно рекомендовать как лучший среди представленных параметров положения -- он показывает стабильный результат даже на таком специфичном распределении, как распределение Коши.

Аномальное поведение данных в распределении Коши, судя по всему, связано с отсутствием у него матожидания (в том смысле, что интеграл, позволяющий его посчитать, расходится).

Хочется отметить полусумму экстремальных значений как наименее стабильный параметр положения: его дисперсия всегда была в разы, а иногда и на несколько порядков, выше, чем у других характеристик. Оно и понятно: мы используем весьма специфичные и уникальные для каждого эксперимента точки -- выбросы распределения -- как опорные элементы для вычисления параметра положения, что, вообще говоря, нелогично.

Все остальные характеристики положения показывали достаточно схожие результаты и их стабильность.

Сравним характеристики для выборок из 1000 элементов по величине:
\begin{itemize}
	\item Нормальное распределение
	
	$\overline{x} \leq z_{tr} \leq med{x} \leq z_Q \leq z_R $
	
	\item Распределение Коши
	
	$z_R \leq \overline{x} \leq med{x} \leq z_{tr} \leq z_Q $
	
	\item Распределение Лапласа
	
	$ \overline{x} \leq z_{tr} \leq med{x} \leq z_Q \leq z_R $
	
	\item Распределение Пуассона
	
	$ z_{tr} \leq med{x} \leq z_Q \leq \overline{x} \leq z_R $
	
	\item Равномерное распределение
	
	$ z_R \leq z_{tr} \leq \overline{x} \leq med{X} \leq z_Q $
	
\end{itemize}

\section{Литература}
Максимов Ю. Д. Математическая статистика //СПб.: СПбГПУ. – 2004.

\section{Приложения}

Репозиторий с кодом программы и кодом отчёта: \href{https://github.com/kystyn/stats}{https://github.com/kystyn/stats}



