\subsection{Выборочные коэффициенты корреляции}

Выборочные коэффициенты корреляции Пирсона и Спирмена достаточно точно описывают истинный коэффициент корреляции двумерной случайной величины: он всегда попадает в доверительный интервал (с центром в соответствующем выборочном коэффициенте и радиусом, равным дисперсии), то есть 1000 вычислительных экспериментов дали правильный результат, при чём вне зависимости от размера выборки. Более точным был всё же коэффициент Пирсона, однако коэффициент Спирмена, как бы знаем, является более универсальным и позволяет оценивать степень коррелированности любых упорядоченных множеств, не обязательно ассоциированных с подмножеством вещественных чисел, что, безусловно, даёт куда большую область применения коэффициента Спирмена. Наше исследование показало его состоятельность.

Квадрантный коэффициент всюду, кроме нулевого коэффициента корреляции давал результат, далёкий от истинного.

\begin{itemize}
\item Для двумерного нормального распределения выборочные коэффициенты корреляции находятся в отношении: $r_S < r_Q \leq r$

\item Для смеси двумерных нормальных распределений -- $r_S < r_Q \leq r$

\end{itemize}
