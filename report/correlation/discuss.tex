Гистограммы в целом визуально повторяют графики плотности распределения, причём чем выше количество случайных величин, тем выше сходство теоретической функции плотности распределения и кривой, огибающей верхние точки столбцов гистограммы (суть, экспериментальной функции плотности распределения).

Однако существуют определённого рода различия: так, максимумы гистограмм и плотностей распределения почти нигде не совпали. Более того, соотношение, кто из них больше, может меняться от эксперимента к эксперименту (распределение Лапласа). Также наблюдаются всплески гистограмм, что наиболее хорошо прослеживается на распределении Коши.
