\subsection{Двумерное нормальное распределение}

Двумерная случайная величина называется нормально распределённой, если её плотность вероятности определена следующим образом:
\begin{multline}
	N(x, y, \overline{x}, \overline{y}, \sigma_x, \sigma_y, \rho) =
	\frac{1}{2 \sigma_x \sigma_y \sqrt{1 - \rho^2}} \times \\	
	\times exp \left( - \frac{1}{2(1 - \rho^2)} \left[ \frac{(x - \overline{x})^2}{\sigma_x^2} - 2 \rho (x - \overline{x}) (y - \overline{y}) + \frac{(y - \overline{y})^2}{\sigma_y^2}) \right] \right) 
\end{multline}

При этом компоненты $X$ и $Y$ также распределены нормально со средним квадратичным отклонением соответственно $sigma_x$ и $sigma_y$.

\subsection{Корреляционный момент и коэффициент корреляции}

\textit{Ковариацией} или \textit{корреляционным моментом} случайной величины называется матожидание произведения отклонений компонент случайной величины от её среднего:
\begin{equation}
	K_{XY} = cov(X, Y) = M[(X - \overline{x})(Y - \overline{y})]
\end{equation}

\textit{Коэффициентом корреляции} является нормированный на единицу корреляционный момент. Показывает меру линейной зависимости между величинами.

\begin{equation}\label{eq:correlation}
	\rho = \frac{K_{XY}}{\sigma_x \sigma_y}
\end{equation}

\subsection{Выборочные коэффициенты корреляции}
\subsubsection{Выборочный коэффициент корреляции Пирсона}

Для выборки двумерной случайной величины $\{x_i, y_i\}_{i=\overline{1,n}}$ наиболее естественным приближением корреляционного коэффициента является соотношение:

\begin{equation}\label{eq:pearson}
	r = \frac{\frac{1}{n} \displaystyle \sum_{i=1}^{n}{\left(x_i - \overline{x}\right)\left(y_i - \overline{y}\right)}}{\sqrt{\displaystyle \sum_{i=1}^{n}{\frac{1}{n}\left(x_i - \overline{x}\right)^2\frac{1}{n}\left(y_i - \overline{y}\right)^2}}} = \frac{K_{XY}}{s_X s_Y},
\end{equation}
где $K, s_X^2, s_Y^2$ -- выборочная ковариация и дисперсии соответствующих случайных величин.

\subsubsection{Выборочный квадрантный коэффициент корреляции}
Альтернативным способом определения взаимосвязи является выборочный квадрантный коэффициент корреляции:

\begin{equation}\label{eq:rq}
	r_Q=\frac{(n_1 + n_3) - (n_2 + n_4)}{n},
\end{equation}

где $n_i$ -- количество точек выборки в соответствующим квадранте координатной плоскости, параллельно перенесённой относительно стандартной на вектор $(med x, med y)$.

\subsubsection{Выборочный коэффициент ранговой корреляции Спирмена}

Как правило, изучаемые на практике объекты обладают некоторым набором качественных признаков, то есть не ассоциированных с конкретными числовыми значениями, но позволяющих задать на множестве объектов отношение полного порядка.

Задав отношение полного порядка, мы тем самым присвоили объектам номера, иначе говоря, проранжировали объекты.

Каждый признак задаёт своё отношение порядка, соответственно, для двух признаков мы будем иметь две последовательности рангов -- $\{u_i\}_{i \in \mathbb{N}}$ и $\{v_i\}_{i \in \mathbb{N}}$.

Коэффициентом ранговой корреляции мы будем называть коэффициент корреляции Пирсона для двумерной выборки $(u_i, v_i)$:

\begin{equation}\label{eq:spearman}
	r_S = \frac{K_{UV}}{s_U s_V}
\end{equation}

\subsection{Эллипсы рассеивания}

Рассмотрим линии уровня плотности вероятности. Они удовлетворяют условию:

\begin{equation}
	\frac{(x - \overline{x})^2}{\sigma_x^2} - 2 \rho (x - \overline{x}) (y - \overline{y}) + \frac{(y - \overline{y})^2}{\sigma_y^2} = const
\end{equation}

Как мы знаем из аналитической геометрии, это уравнение эллипса.

Данное уравнение представляет из себя квадратичную форму, приведя которую к каноническому виду и отнормировав на константу, стоящую справа от знака равенства, мы получим каноническое уравнение эллипса.

Матрица преобразований квадратичной формы подскажет нам расположение системы координат, в которой фокусы эллипса будут лежать на абсциссе и, соответственно, смещение и угол поворота этой системы координат относительно старой. Таким образом, мы узнаем, как направлены полуоси данного эллипса относительно исходной системы координат.

Произведя соответствующие выкладки, получаем, что большая полуось эллипса составляет с абсциссой угол, выраженный по следующей формуле:
\begin{equation}
	tg 2\alpha = \frac{2 \rho \sigma_x \sigma_y}{\sigma_x^2 - \sigma_y^2}
\end{equation}

Посмотрев на данное соотношение, можно понять, что угол поворота эллипса даёт нам качественное представление о степени коррелированности данных.

Контур эллипса является линией равной вероятности, поэтому такой эллипс называется эллипсом равной плотности либо эллипсом рассеивания.