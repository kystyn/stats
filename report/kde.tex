\section{Постановка задачи}
Даны пять распределений:
\begin{itemize}
\item Нормальное распределение: $N(x, 0, 1)$
\item Распределение Коши: $C(x, 0, 1)$
\item Распределение Лапласа: $L(x, 0, \frac{1}{\sqrt{2}})$
\item Распределение Пуассона: $P(k, 10)$
\item Равномерное распределение: $U(x, -\sqrt{3}, \sqrt{3})$
\end{itemize}

Для каждого распределения требуется сгенерировать выборки из 20, 60 и 100 элементов.

Построить на них эмпирические функции распределения и ядерные оценки плотности распределения:
\begin{itemize}
	\item Для непрерывных распределений -- на отрезке $[-4; 4]$
	\item Для распределения Пуассона -- на отрезке $[6; 14]$
\end{itemize} .

Для каждого распределения требуется:
\begin{itemize}
	\item Определить долю выбросов, сгенерировав выборку, соответствующую распределению, 1000 раз и вычислив среднюю долю выбросов
	\item Сравнить с результатами, полученными теоретически
\end{itemize}

\section{Теория}

\subsection{Эмперическая функция распределения}
\subsubsection{Статистический ряд}	Статистический ряд -- это упорядоченная по возрастанию последовательность \textit{различных} элементов выборки $\{z_i\}_{i = \overline{1,k}}$ с  указанием частот $\{n_i\}_{i = \overline{1,k}}$, с которыми элементы с соответствующим индексом встречаются в исходной выборке.

\begin{tabular}{|c|c|c|c|c|}
	\hline
	$z$ & $z_1$ & $z_2$ & $\ldots$ & $z_k$ \\
	\hline
	$n$ & $n_1$ & $n_2$ & $\ldots$ & $n_k$ \\
	\hline
\end{tabular}

\subsubsection{Определение}
Эмпирической функцией распределения называется функция, которая заданному вещественному числу $x$ сопоставляет относительную частоту события $X \leq x$, полученную по данной выборке:

\begin{equation}\label{edf}
F^*(x)=P^*(X \leq x)
\end{equation}

\subsubsection{Вычисление}
Для того чтобы посчитать эмпирическую функцию вероятности в заданной точке, можно построить статистический рад и просуммировать частоты, с которыми встречаются все элементы, меньшие, чем $x$  -- таким образом мы получим количество возможных событий $X \leq x $. Для вычисления относительной частоты остаётся разделить полученное значение на общее количество событий.

\begin{equation}\label{edf}
F^*(x)=P^*(X \leq x) = \displaystyle \frac{1}{n}\sum_{z_i \leq x}{n_i}
\end{equation}

где $n$ -- число событий

\subsection{Оценки плотности вероятности}
\subsubsection{Определение}

Оценкой плотности вероятности называется функция $\hat{f}(x)$, построенная по выборке и приближённо равная плотности вероятности:

\begin{equation}
\hat{f}(x) \approx f(x)
\end{equation}

\subsubsection{Ядерные оценки}

Будем оценивать плотность вероятности следующим образом:

\begin{equation}
\hat{f_n}(x) = \displaystyle \frac{1}{nh_n}\sum_{i=1}^{n}{K\left(\frac{x - x_i}{h_n}\right)}
\end{equation}

Здесь $K(u)$ -- ядро: функция, обладающая свойствами:
\begin{enumerate}
	\item \begin{equation}K \in L^1 \end{equation}
	\item \begin{equation}K(u) \geq 0 \end{equation}   
	\item \begin{equation}|K\|_{L^1}=1\end{equation} \label{kde_norm}
\end{enumerate}

Эти требования также будут выполнены и для описанной суммы: 1 -- поскольку $L^1$ -- линейное пространство, 2 -- очевидно, 3 -- по свойствам нормы. Таким образом, $\hat{f}$ удовлетворяет определению плотности вероятности.

Число $h_n$ называется шириной полосы пропускания.
Последовательность $\{h_n\}_{n \in \mathbb{N}}$ должна удовлетворять следующим условиям:
\begin{enumerate}
\item 
	\begin{equation}
	h_n \underset{n \rightarrow \infty}{\longrightarrow} 0
	\end{equation}
\item 
	\begin{equation}
	\frac{h_n}{n^{-1}} \underset{n \rightarrow \infty}{\longrightarrow} \infty
	\end{equation}
\end{enumerate}

При выполнении всех условий данные оценки называются ядерными.

Ядерная оценка является состоятельной в том смысле, что сходится по распределению к плотности вероятности случайной величины ([2, стр. 38, ф-ла 4]).

В качестве ядра будем использовать нормальное ядро:
\begin{equation}
K(u)=\frac{1}{\sqrt{2\pi}}e^{-\frac{u^2}{2}}
\end{equation}

Для вычисления ширины полосы пропускания будем использовать эмпирическое правило Сильвермана:
\begin{equation}
h_n=1.06\hat{\sigma}n^{-1/5},
\end{equation}
где $\hat{\sigma}$ -- выборочное стандартное отклонение.
	
\section{Реализация}
Данная работа реализована на языке программирования Python с использованием IDE PyCharm и библиотек NumPy, MatPlotLib, Seaborn, SciPy, StatsModel в ОС Ubuntu 19.04.

Отчёт подготовлен с помощью компилятора pdflatex и среды разработки TeXStudio.

\section{Результаты}
\subsection{Эмпирическая функция распределения}

\begin{figure}[H]
	\begin{center}
		\includegraphics[scale=0.5]{cdf/Normal_cdf}
		\caption{Нормальное распределение} 
		\label{pic:pic_name} 
	\end{center}
\end{figure}

\begin{figure}[H]
	\begin{center}
		\includegraphics[scale=0.5]{cdf/Cauchy_cdf}
		\caption{Распределение Коши} 
		\label{pic:pic_name} 
	\end{center}
\end{figure}

\begin{figure}[H]
	\begin{center}
		\includegraphics[scale=0.5]{cdf/Laplace_cdf}
		\caption{Распределение Лапласа} 
		\label{pic:pic_name} 
	\end{center}
\end{figure}

\begin{figure}[H]
	\begin{center}
		\includegraphics[scale=0.5]{cdf/Poisson_cdf}
		\caption{Распределение Пуассона} 
		\label{pic:pic_name} 
	\end{center}
\end{figure}

\begin{figure}[H]
	\begin{center}
		\includegraphics[scale=0.5]{cdf/Uniform_cdf}
		\caption{Равномерное распределение} 
		\label{pic:pic_name} 
	\end{center}
\end{figure}


\subsection{Ядерные оценки плотности распределения}

\begin{figure}[H]
	\begin{center}
		\includegraphics[scale=0.5]{kde/Normal20_kde}
		\caption{Нормальное распределение, $n=20$} 
		\label{pic:pic_name} 
	\end{center}
\end{figure}

\begin{figure}[H]
	\begin{center}
		\includegraphics[scale=0.5]{kde/Normal60_kde}
		\caption{Нормальное распределение, $n=60$} 
		\label{pic:pic_name} 
	\end{center}
\end{figure}

\begin{figure}[H]
	\begin{center}
		\includegraphics[scale=0.5]{kde/Normal100_kde}
		\caption{Нормальное распределение, $n=100$} 
		\label{pic:pic_name} 
	\end{center}
\end{figure}

\begin{figure}[H]
	\begin{center}
		\includegraphics[scale=0.5]{kde/Cauchy20_kde}
		\caption{Распределение Коши, $n=20$} 
		\label{pic:pic_name} 
	\end{center}
\end{figure}

\begin{figure}[H]
	\begin{center}
		\includegraphics[scale=0.5]{kde/Cauchy60_kde}
		\caption{Распределение Коши, $n=60$} 
		\label{pic:pic_name} 
	\end{center}
\end{figure}

\begin{figure}[H]
	\begin{center}
		\includegraphics[scale=0.5]{kde/Cauchy100_kde}
		\caption{Распределение Коши, $n=100$} 
		\label{pic:pic_name} 
	\end{center}
\end{figure}

\begin{figure}[H]
	\begin{center}
		\includegraphics[scale=0.5]{kde/Laplace20_kde}
		\caption{Распределение Лапласа, $n=20$} 
		\label{pic:pic_name} 
	\end{center}
\end{figure}

\begin{figure}[H]
	\begin{center}
		\includegraphics[scale=0.5]{kde/Laplace60_kde}
		\caption{Распределение Лапласа, $n=60$} 
		\label{pic:pic_name} 
	\end{center}
\end{figure}

\begin{figure}[H]
	\begin{center}
		\includegraphics[scale=0.5]{kde/Laplace100_kde}
		\caption{Распределение Лапласа, $n=100$} 
		\label{pic:pic_name} 
	\end{center}
\end{figure}

\begin{figure}[H]
	\begin{center}
		\includegraphics[scale=0.5]{kde/Poisson20_kde}
		\caption{Распределение Пуассона, $n=20$} 
		\label{pic:pic_name} 
	\end{center}
\end{figure}

\begin{figure}[H]
	\begin{center}
		\includegraphics[scale=0.5]{kde/Poisson60_kde}
		\caption{Распределение Пуассона, $n=60$} 
		\label{pic:pic_name} 
	\end{center}
\end{figure}

\begin{figure}[H]
	\begin{center}
		\includegraphics[scale=0.5]{kde/Poisson100_kde}
		\caption{Распределение Пуассона, $n=100$} 
		\label{pic:pic_name} 
	\end{center}
\end{figure}

\begin{figure}[H]
	\begin{center}
		\includegraphics[scale=0.5]{kde/Uniform20_kde}
		\caption{Равномерное распределение, $n=20$} 
		\label{pic:pic_name} 
	\end{center}
\end{figure}

\begin{figure}[H]
	\begin{center}
		\includegraphics[scale=0.5]{kde/Uniform60_kde}
		\caption{Равномерное распределение, $n=60$} 
		\label{pic:pic_name} 
	\end{center}
\end{figure}

\begin{figure}[H]
	\begin{center}
		\includegraphics[scale=0.5]{kde/Uniform100_kde}
		\caption{Равномерное распределение, $n=100$} 
		\label{pic:pic_name} 
	\end{center}
\end{figure}

\section{Обсуждение}

Эмпирическая функция распределения на всех распределениях достаточно точно приближается к истинной функции распределения, причём это свойство сохраняется даже на выборке из двадцати элементов: оценивая на глаз, можно сказать, что точечные отличия для непрерывных распределений не превосходят 10\%, а для распределения Пуассона достигли 20\% на выборке из 
 элементов.
 
 Более интересно дела обстоят с ядерной оценкой плотности: эмпирическая ширина полосы пропускания Сильвермана оказалась оптимальной не везде: так, у распределения Лапласа при всех тестируемых размерах выборок ядерная оценка получилась куда точнее, когда полоса пропускания была в два раза уже, чем оптимальная
 
 В целом ядерная оценка получилась наиболее гладкой в смысле количества экстремальных точек при интервале пропускания, чья ширина в два раза больше оптимального. То есть оптимальная оценка иногда, быть может, обеспечивает худшую поточечную сходимость, но лучшую интегральную ($L^p$).
 
 Стоит отметить, что ядерная оценка дала очень негладкое приближение распределения Пуассона при половинном от оптимального интервале пропускания. При оставшихся интервалах пропускания результаты получились более удовлетворительными в смысле гладкости, но всё ещё очень неточными в смысле интегральной метрики до истинной плотности распределения.
 Вероятно, это обосновано принципиально дискретным характером распределения, что осложняет сходимость на малых размерах выборки.
\section{Литература}
[1] Максимов Ю. Д. Математическая статистика //СПб.: СПбГПУ. – 2004.

[2]
\href{http://quantile.ru/07/07-SA.pdf}{Анатольев Станислав (2009) "Непараметрическая регрессия", Квантиль, №7, стр. 37-52} \href{https://ru.wikipedia.org/wiki/%D0%92%D1%8B%D0%B1%D0%BE%D1%80%D0%BE%D1%87%D0%BD%D0%BE%D0%B5_%D1%81%D1%80%D0%B5%D0%B4%D0%BD%D0%B5%D0%B5}{Википедия: выборочное среднее}

\section{Приложения}

Репозиторий с кодом программы и кодом отчёта: \href{https://github.com/kystyn/stats}{https://github.com/kystyn/stats}


