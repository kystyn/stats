\subsection{Проверка гипотезы о законе генеральной совокупности. Метод $\chi^2$}

\subsubsection{Метод максимального правдоподобия}

Для сгенерированной выборки были получены результаты:

$$\hat{\mu}=0.007, \hat{\sigma}=1.05$$

\subsubsection{Критерий согласия $\chi^2$}

\begin{table}[H]
	\begin{center}
		\begin{tabular}{|c|c|c|c|c|c|c|}
\hline i & Границы $\Delta_i$ & $n_i$ & $p_i$ & $np_i$ & $n_i - np_i$ & $\frac{(n_i - np_i)^2}{np_i}$\\\hline
1 & -1.72, -0.52 & 0.00 & 0.0010 & 0.10 & -0.10 & 0.10\\
\hline
2 & -0.52, 0.68 & 4.00 & 0.0304 & 3.04 & 0.96 & 0.31\\
\hline
3 & 0.68, 1.88 & 26.00 & 0.2361 & 23.61 & 2.39 & 0.24\\
\hline
4 & 1.88, 3.08 & 42.00 & 0.4652 & 46.52 & -4.52 & 0.44\\
\hline
5 & 3.08, -2.92 & 24.00 & 0.2361 & 23.61 & 0.39 & 0.01\\
\hline
6 & -2.92, -1.72 & 4.00 & 0.0304 & 3.04 & 0.96 & 0.31\\
\hline
7 & -1.72, -0.52 & 0.00 & 0.0010 & 0.10 & -0.10 & 0.10\\
\hline
$\Sigma$ & & 100 & 1.0000 & 100.00 & 0.00 & $\chi_{\text{В}}^2=1.49$ \\
\hline
\end{tabular}


		\caption{Вычисление $\chi_{\text{В}}^2$ при проверке закона о нормальном распределении для выборки нормального распределения}
	\end{center}
\end{table}

Видим, что $\chi_{\text{В}}^2 < \chi_{0.95}^2(6)$, следовательно, гипотезу полагаем верной.

\begin{table}[H]
	\begin{center}
		\documentclass{article}
\usepackage{amsmath}
\usepackage[english,russian]{babel}

\begin{document}
\begin{table}
\begin{center}
\begin{tabular}{|c|c|c|c|c|c|c|}
\hline i & Границы $\Delta_i$ & $n_i$ & $p_i$ & $np_i$ & $n_i - np_i$ & $\frac{(n_i - np_i)^2}{np_i}$\\\hline
1 & -1.89, -0.69 & 0.00 & 0.0000 & 0.00 & -0.00 & 0.00\\
\hline
2 & -0.69, 0.51 & 0.00 & 0.0071 & 0.18 & -0.18 & 0.18\\
\hline
3 & 0.51, 1.71 & 7.00 & 0.1998 & 4.99 & 2.01 & 0.81\\
\hline
4 & 1.71, 2.91 & 13.00 & 0.5863 & 14.66 & -1.66 & 0.19\\
\hline
5 & 2.91, -3.09 & 5.00 & 0.1998 & 4.99 & 0.01 & 0.00\\
\hline
6 & -3.09, -1.89 & 0.00 & 0.0071 & 0.18 & -0.18 & 0.18\\
\hline
7 & -1.89, -0.69 & 0.00 & 0.0000 & 0.00 & -0.00 & 0.00\\
\hline
$\Sigma$ & & 25 & 1.0000 & 25.00 & 0.00 & 1.35 \\
\hline
\end{tabular}
\caption{Распределение Лапласа}
\end{center}
\end{table}

Получили, что $\chi_{\text{В}}^2 < \chi_{0.95}^2(6)$, следовательно гипотеза о нормальности принимается.
\end{document}

		\caption{Вычисление $\chi_{\text{В}}^2$ при проверке закона о нормальном распределении
		для выборки распределения Лапласа}
	\end{center}
\end{table}

Получили, что $\chi_{\text{В}}^2 < \chi_{0.95}^2(6)$, следовательно гипотеза о нормальности принимается.