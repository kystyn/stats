\section{Обсуждение}\label{discuss}
\subsection{Зависимость расстояния между экстремумами от времени, оставшегося до срыва разряда}

В четырёх из девяти экспериментов действительно наблюдается тенденция роста построенной величины по мере приближения к срыву разряда. Однако только в двух экспериментах этот рост произошёл в самом конце. Это означает, что, наблюдая только рост данной величины, нельзя предсказать момент срыва разряда. Тем не менее, есть и позитивный результат: примерно в половине случаев можно говорить, что если идёт снижение частоты, значит срыва разряда в ближайшее время точно не будет.

Кардинально выбиваются из данных соображений только эксперименты sht38852, sht38867 и sht38877.

\subsection{Графики локальных минимумов частотных портретов. Время от момента достижения <<дна>> частотной истории до срыва}

По графикам локальных минимумов видно, что <<дно>> частотной истории в реальном времени получается определять в 6 из 9 экспериментов. Примечательно, что четыре из них совпадают с четырьмя, удовлетворившими \hyperref[analysis]{предыдущей гипотезе}.

\subsection{Основные результаты}

Время от момента достижения <<дна>> частотной истории до момента срыва, как видно из таблицы ~\ref{table:dt}, имеет относительную погрешность 50\%, если рассматривать его по всем экспериментам\footnote{Рассмотрен только 18 датчик ввиду того, что во всех экспериментах, где проверяемые гипотезы подтверждаются, все датчики показывают практически одинаковые частотные портреты}.

Однако если посмотреть этот же параметр только по датчикам, удовлетворившим \hyperref[globalMin]{замечанию} ко второму параметру (см. ~\ref{pic:min2}), то дисперсия станет немного лучше -- уже 0.09, вместо 0.1, что позволяет оценивать второй знак после запятой -- и относительная погрешность составляет уже порядка 35\%. 

Если же смотреть только те эксперименты, которые удовлетворили критерию возрастания \hyperref[eq:wDist]{взвешенного расстояния} между экстремумами (см. ~\ref{pic:min3}), то дисперсия получается уже совсем небольшой, 0.03, и относительная погрешность составляет около 13\%.

Таким образом, получена следующая закономерность: если достигается дно частотной истории и после этого события начинает расти взвешенное расстояние между экстремумами, то в среднем через 0.023 секунды случается срыв разряда.