\section{Постановка задачи}

В процессе проведения нескольких экспериментов были получены экспериментальные данные с разных датчиков, регистрирующих рентгеновское излучение плазмы.

Эксперимент заканчивался через некоторое время после того, как происходил срыв плазмы. Было визуально установлено, что срыву предшествует участок, на котором наблюдаются пилообразные колебания регистрируемого напряжения.

После построения частотного портрета всех участков пилообразных колебаний было сделано наблюдение: срыв разряда обычно происходит в конце затяжного участка возрастания частоты, берущего начало на <<дне>> частотной истории.

Требуется определить статистическую взаимосвязь между этими явлениями.

\section{Анализ данных}\label{analysis}

Было замечено, что частотный портрет данных имеет периодический характер. Замечено также, что во многих экспериментах срыв разряда случается после того, как закончится участок возрастания частоты, длящийся значительно б\'{о}льшее время, чем предыдущие, и обеспечивающий б\'{о}льший рост частоты. Кроме того, такой участок  начинается из точки глобального минимума всего графика.

В связи с этим, представляет интерес исследование следующих параметров:
\begin{enumerate}
	\item \label{extremumDist} Зависимость расстояния между экстремумами от времени, оставшегося до срыва разряда
	
	\item \label{timeBottom} Время, прошедшее с момента достижения <<дна>> частотной истории, до срыва
\end{enumerate}

%\begin{remark}
%	Критерии, сопоставляющие исследуемые параметры и прогноз срыва, будем нумеровать так же, как и сами параметры -- первый и второй.
%\end{remark}

\hyperref[extremumDist]{Первое соотношение} призвано дать прогноз момента срыва на основании периода колебаний частоты. 

Евклидово расстояние между двумя точками будет не информативно: оно будет почти равно $\mathbb{R}^1$ метрике ввиду отличия порядков величин, отложенных по координатным осям.

Поэтому мы будем анализировать следующее выражение:

\begin{equation}\label{eq:wDist}
dist(\{t_1, f_1\}, \{t_2, f_2\}) = \sqrt{((t2 - t1) \cdot C_{TF}) ^2 + (f_2 - f_1) ^ 2},
\end{equation}

где $C_{TF}$ -- константа, которую мы будем называть \textit{весом}.

Вес вносится, чтобы сравнять <<роли>> времени и частоты и нужен, поскольку величина частоты отличается от величины времени на несколько (в данной серии экспериментов -- четыре) порядков. Поэтому положим $C_{TF} = 10000$. Будем называть далее такую величину \textit{взвешенным расстоянием}.

\begin{remark}\label{globalMin}
Важно заметить, что \hyperref[timeBottom]{второй параметр} имеет практический смысл, только если есть понимание того, как обнаружить <<дно>> в реальном времени.
\end{remark}

Здесь подразумевается, что глобальный минимум должен определяться однозначно. Это означает, что \textit{в последовательности минимумов} должен быть ровно один локальный минимум (он же -- глобальный). Иными словами, видя, что последовательность минимумов убывает, наблюдатель должен понимать, что дно не достигнуто, но как только находится локальный минимум, больший предыдущего, так сразу должно быть понятно что дно пройдено.