\section{Постановка задачи}

В процессе проведения нескольких экспериментов были получены экспериментальные данные с разных датчиков, регистрирующих рентгеновское излучение плазмы.

Эксперимент заканчивался через некоторое время после того, как происходил срыв плазмы. Было визуально установлено, что срыву предшествует участок, на котором наблюдаются пилообразные колебания регистрируемого напряжения.

После построения частотного портрета всех участков пилообразных колебаний было сделано наблюдение: перед срывом разряда наблюдается участок снижения частоты.

Требуется определить статистическую взаимосвязь между участками снижения частоты и временем конца разряда.

\section{Анализ данных}\label{analysis}

Было замечено, что частотный портрет данных имеет периодический характер. Замечено также, что во многих экспериментах срыв разряда случается после того, как закончится участок возрастания частоты, длящийся значительно б\'{о}льшее время, чем предыдущие, и обеспечивающий б\'{о}льший рост частоты. Кроме того, такой участок  начинается из точки глобального минимума всего графика.

В связи с этим, представляет интерес исследование следующих параметров:
\begin{itemize}
	\item Зависимость расстояния между экстремумами от времени, оставшегося до срыва разряда
	
	\item Время, прошедшее с момента достижения <<дна>> частотной истории, до срыва
\end{itemize}

Первое соотношение призвано дать прогноз момента срыва на основании периода колебаний частоты. Под расстоянием между экстремумами мы понимаем величину вида:

\begin{equation}
dist(\{t_1, f_1\}, \{t_2, f_2\}) = \sqrt{((t2 - t1) * 10000) ^2 + (f_2 - f_1) ^ 2}
\end{equation}

Вес 10000 вносится, чтобы сравнять <<роли>> времени и частоты и нужен, поскольку величина частоты отличается от величины времени на четыре порядка. Будем называть далее такую величину \textit{взвешенным расстоянием между экстремумами}.

Важно заметить, что второй критерий имеет практический смысл, только есть если понимание того, как обнаружить <<дно>> в реальном времени.

Здесь подразумевается, что глобальный минимум должен определяться однозначно. Это означает, что \textit{в последовательности минимумов} должен быть ровно один минимум (он же -- глобальный). Иными словами, видя, что последовательность минимумов убывает, наблюдатель должен понимать, что дно не достигнуто, но как только находится минимум, больший предыдущего, так сразу должно быть понятно что дно пройдено.